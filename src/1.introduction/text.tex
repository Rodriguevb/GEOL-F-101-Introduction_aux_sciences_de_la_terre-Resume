GEOL-F-101 - Introduction aux sciences de la terre
Résumé modifiable :
https://docs.google.com/document/d/1TdnjWUnvqybxBER-TZ_dV8E7kcK6pffdi8IjBmuvCw0









Aide à l’étude

Introduction aux sciences de la terre

GEOL-F-101
________________


Chapitre 1: Le système terre
Âge de la terre: 4,6 milliard d’année.


1.1 La terre en tant que système:
4 entitées:
- L’atmosphère: fine enveloppe de gaz qui entoure la terre.
- L’hydrosphère: Différent réservoir d’eau ( Y compris la glace)
- La biosphère: Les organismes vivants
- La terre solide: Inclus toutes les roches


  

	



3 parties :
1. Noyau
2. Manteau : épaisse couche de roche.
3. Croûte : fine péllicule externe.
	

Forçages: influences internes ou externes sur les différentes entités du système terre.


1.2 Courbe de Keeling:
Concentration CO2 : 315 ppmv (1958) à 387.8 ppmv (2008)
De octobre à mai, la concentration va augmenter et de mai à octobre, elle va diminuer.
Pourquoi ? Photosynthèse et respiration des forêts.
        CO2 + H2O ⇔  CH2O + O2
        CO2 faible en automne car photosynthèse estivale.
        CO2 élevé au printemps suite a la respiration hivernale.
Taux d’accroissement annuel ± 1.4 ppm/an ( 0.6 dans les 60’ et 1.9 maintenant)


1.3 Enregistrements:
La composition de l’atmosphère dans le passé (récent) peut être obtenu en analysant la composition des bulles d’air piégées dans la glace polaire.


L’accroissement en CO2 a commencé au 19ème siècle à la suite de la déforestation en Amérique du nord. La concentration était d’environ 280 ppmv
L’activité humaine est donc responsable d’un accroissement d’environ 40 % en CO2 atmosphérique au cours des deux derniers siècles.




1.4 D’autres gaz à effets de serre:
En plus du CO2, il y a aussi le méthane (CH4) et l’oxyde nitreux (N2O).
Le CH4 est passé de 0,7 à 1,8 ppm => 18x plus puissant que CO2.


Clathrates: plusieurs molécules d’eau qui emprisonne du CH4 => Très instable !




La température a augmenté de 0,8°C au cours du 20ème siècle.
Elle n’augmente pas de manière aussi uniforme que le CO2 => D’autres facteurs.


De 1940 à 1970: Ralentissement ou arrêt du réchauffement.
Cause possible: Réflection accrue du rayonnement solaire par des aérosol de sulfate formés à partir de l’oxydation du dioxyde de souffre émis lors de la combustion du charbon. (Maintenant éliminé pour limiter les pluies acides.)




  
1.5 Changement globaux: Long-terme
Échelle des temps géologiques:
- Les éons : Le niveau hiérarchique le plus élevé.
- Les ères : subdivision des éons.
- Les périodes : subdivisions des ères.
- Les époques : subdivision des périodes.






1.6 Cycles glaciaires et interglaciaires:


Dans une carotte de glace, on analyse la quantité de CO2 et la quantité de Deutérium qui peut être utilisé comme ‘proxy’ pour la température.


Le dernier âge glaciaire s’est terminé il y a ± 11.000 ans.


Pourquoi les concentrations ont augmentés de manière si abrupte il y a 140.000  et 21.000 ans ?
Une cause possible: Changement de la circulation générale océanique. L’océan contient de grandes quantités de CO2 dissous en partie relarguée dans l’atmosphère lorsque ces eaux sont ramenés en surface. La circulation océanique depend du climat.
Donc les concentrations de CO2 atmosphérique affecte le climat, mais le climat affecte les niveaux de CO2 atmosphérique.


1.7 Extinctions de masse:
Il y a 65 millions d’années, à la fin de l’ère mésozoïque, les dinosaures, 60 à 80% des espèces marine, ainsi qu’un grand nombre de plantes et d’animaux terrestres ont disparus de manière abrupte.


Limite K-T: transition entre la période Crétacée et la période Tertiaire.


L'iridium dans les roches provient exclusivement de petit débris de matière extraterrestre retombant de manière continue sur terre.


Dans de l’argile datant de la limite K-T, l’abondance en irridium atteint des valeurs de 10 ppm par unité de masse. Il doit venir d’un objet tel un astéroïde.
La masse estimée est de 1015 kg ( c’est à dire 10 km de diamètre)
=> Ensuite, découverte d’un cratère de 200 km dans la région de Chicxulub. ( Mexique)


Variation de luminosité solaire:
L’énergie solaire provient de la fusion nucléaire.
Il y a 4,6 milliard d’année, le soleil était 30% moins lumineux.
Aujourd’hui elle augmente de 1 %/100 millions d’années.
A la fin de sa vie ( dans 5 milliards d’années), sa brillance devrait être 2 à 3 fois plus élevée que maintenant.






Si le soleil était si peu lumineux, les océans auraient du être couvert de glace jusqu’à il y a 2 milliards d’années. Hors, l’eau liquide était déjà présente il y a 3,8 milliards d’années.
=> “Paradoxe du jeune soleil”


La solution la plus probable: Niveaux des gaz à effets de serre était significativement plus élevés qu'aujourd'hui.


Question: Le système climatique sur terre aurait-il un méchanisme propre lui permettant de maintenir la température dans un domaine ou la vie est possible ?






Chapitre 2: Planète Terre: Balance énergétique  et l’effet de serre.


Vie sur terre due à son climat tempéré.
Présence d’eau indispensable à la vie.


La terre doit sa température à la distance qui la sépare du soleil et à l’effet de serre. Sans celui-ci, la température serait de -18°C.


2.1 Transport d’énergie:
3 mécanismes fondamentaux:
- Conduction
- Convection
- Rayonnement
	  

	





2.2 Rayonnement:
Énergie transférée par des ondes électromagnétiques.
Une onde est définie par 3 caractéristiques fondamentales:
- Longueur d’onde: Longueur d’un cycle complet (prononcé LAMBDA)
- Fréquence: 
- Vitesse : 
2.3 Énergie d’une onde:
L’énergie (E) est proportionelle à la fréquence et inversement proportionelle à la longueur d’onde:

constante de Planck (6,63 .10-34 J.s)
vitesse de la lumière (m.s-1)
longueur d’onde (m)


L’énergie d’une onde influe sur la manière dont elle interagit avec la matière:
Haute énergie : Peut briser des molécules
Faible énergie: Affecte seulement les vitesses de rotation ou l’intensité des vibrations des molécules.
iu 
1000 µm > LAMBDA > 1 = Infra-rouge
1 µm > LAMBDA > 0.1 = Lumière visible
0.1 µm > LAMBDA > 0.01 = Ultra-violet


Rayonnement solaire = 99,9% du flux total d’énergie sur terre


~50% de l’énergie dans la région visible
~40% dans proche-IR
~10% dans UV


Lois fondamentales du rayonnement:
1) Tous les objets émettent de l’énergie radiative
2) La quantité totale d’énergie radiative émise est prop. à la quatrième puissance de la température de l’objet:
        Loi de Stefan Boltzmann
        F = SIGMA T4
        où SIGMA = 5,67 * 10-8 ( cst)
3) Au plus chaud est l’objet au plus court est la longueur d’onde.
        Loi de Wien
        LAMDAmax = 3000/T






Balance énergétique de la planète terre:
Aire d’une sphère = 4 pi r²


La terre est à l’état stationnaire: Sa température est constante
=> La quantité d’énergie émise par la terre est égale à l’énergie reçue.
Energie reçue:
S0 = L / ( 4 pi rs-t²) ( constante solaire )
=> Pour la terre : S0 = 1370 W/m²
Energie recue = S0 * pi rt²
Mais Albedo ( 30% énergie réfléchie) = > A = 0,3
Ein = S0 pi re² (1-A) = S0 * pi * re² * 0,7


Energie émise
Eout = F * surface terre
        F = sigma * T4
        Surface = 4 Pi Re²
=> Eout = (sigma * T4) * ( 4 pi re² )


Equilibre radiatif:
Ein = Eout
=> T4 = (S0 * ( 1 - A )) / ( 4 * sigma)
On peut donc calculer la température de la terre si elle se comportait comme un corps noir.
T4 = 4,23 * 109
Texp= 255 K = -18°C
Hors la température moyenne est de 15°
=> Diférence de 33° du à l’effet de serre.


Gaz à effet de serre:
Composition de l’atmosphère:


Gaz
	concentration (%)
	(ppm)
	N2
	78
	

	O2
	21
	

	Ar
	0,9
	

	H2O
	variable
	0,1-40.000
	CO2
	0,037
	370
	CH4
	

	1,7
	N2O
	

	0,3
	O3
	

	1,0 à 0,01
	Constituants majeurs
Constituants mineurs ( gaz à effets de serres )


Les gaz à effet de serre ont un moment dipolaire alors que les molécules N2 et O2 n’en ont pas)
L’énergie reçue accroît le mouvement des molécules et perturbe leur vitesse de rotation ou l’intensité de leur vibration.
  





Chapitre 3 : La terre dynamique:
Système de circulation globaux.


Le système terre est en mouvement permanent.
Des mouvement de circulations globaux répartissent l’énergie pour tendre vers un état d’équilibre dynamique. ( ex: vent, plaques tectoniques, …)


Les mouvements circulatoires globaux régule la température du globe:
- Dans la terre fluide, redistribuent l’énergie solaire
- Dans la terre solide, régulent les niveaux de CO2 atmosphérique et la température.
Ils sont conditionnés par différentes pompes.


Les pompes:
- Échelles de temps courtes (1-10 ans): Pompe la plus importante = Océans tropicaux
        => Mouvements atmosphériques et courants marins de surface.,
        Source d’énergie = Soleil
- plus longues ( 1000-10000 ans) : Une deuxième pompe contrôle la circulation océanique profonde, Source d’énergie = Soleil
-Les plus longues (> 106 ans) Mouvement continent,
        Source d’énergie = Radioactivité dans le globe terrestre.


Ce n’est que grâce à ces systèmes circulatoires que le système Terre se maintient dans un état de relative stabilité.
 


Cycle de l’énergie: Distribution spatiale d’énergie radiative
Température moyenne globale : Déterminée par l’équilibre radiatif entre énergie solaire absorbée et énergie IR émise vers l’espace par la Terre.
=> L’énergie émise et l’énergie reçue ne sont pas distribuées de manière uniforme.


  



  



L’excès d’énergie est transférée vers les pôles par la convection.
La différence de température initie des mouvements dans l’atmosphère, puis à la surface des océans.


La terre fluide (1) : Circulation atmosphérique
Structure de l’atmosphère
-Composition (chap II)
-Pression: 1013 bar au niveau de la mer.
-Température


La pression décroit exponentiellement en fonction de l’altitude.
90% de la masse de l’atm sous 16km
50% de la masse de l’atm sous 5.6km


  





Troposphère:
-Détermine la météo
-Chaude à la base, plus froide dans la région supérieure
-INSTABLE
-Circulation par convection
-Eau est importante


Stratosphère:
-Froide à la base, plus chaude dans la région supérieure
-STABLE
-Non convective




Convection: Processus par lequel la chaleur est transportée par le mouvement d’un fluide.
Conduction : Chaleur transférée par contact direct entre les molécules.




La température contrôle la circulation atmosphérique ( dû  à la relation inverse entre T° et densité)


Ascension de masses d’air de plus faibles densités.
  

Complication: H2O
Chaleur latente: énergie nécessaire pour réaliser un transfert de phase.


La circulation atmosphérique transporte de la chaleur latente vers les pôles sous forme de vapeur d’eau.


Complication: La rotation de la terre
La rotation de la terre conduit à l’effet de Coriolis.
La force de Coriolis dévie les vent:
-Vers la droite dans l’hémisphère nord
-Vers la gauche dans l’hémisphère sud
=> Les deux grandes cellules de circulation sont brisées en plusieurs petite cellules.
  



La Terre fluide (2): Le cycle hydrologique
70% de la surface de la Terre est recouverte par les océans.
L’eau est également abondamment présente dans l’atmosphère
L’eau est la seule substance existant naturellement sous forme solide, liquide et gazeuse aux condition de T° et P de la surface terrestre.
Elle constitue l’agent principal par lequel l’énergie et la matière circulent entre les différents compartiments du système Terre.


  





Réservoirs:
-Océan (97%): Eau salée
-Continents(3%):Glace polaire(>2%),glacier,neige,lacs,rivières et eau souterraines.
 -Atmosphère(<0.001%):Vapeur d’eau et nuage.


Les réservoirs et les mouvements d’eau entre ceux-ci caractérisent ce que l’on appelle le cycle hydrologique.


  





Circulation océanique de surface:
La surface des océans est chauffée par le rayonnement solaire, ce qui conduit à une situation dynamique stable=>L’effet de température à lui seul ne peut conduire à un phénomène de convection océanique.
=> Cause: Distribution globale eds vents par éffet de friction


  







Circulation océanique profonde:
Résulte de différence de densité.


Facteur de la densité ?
-Température
-Salinité ( contenu en sel d’une masse d’eau )


Les sels contenu dans l’eau de mer proviennent principalement de l’altération des roches crustales (roches de la croûte terrestre ) => Processus chimique et physique.




Élimination du sel de l’eau de mer:
1. Évaporation d’eau dans les mers peu profondes. Les sels se concentres jusqu'à ce qu’ils précipitent sous forme de dépôts d’évaporites.


2. Processus biologique: Certains organismes éliminent les élément chimique tels le Calcium ou le silicium pour former leurs coquilles.


3. Réaction chimiques entre l’eau de mer et les roches volcaniques se formant sur les fonds océanique.
4. Formation d’aérosol marins. De petites gouttes d’eau de mer sont emportée par l’atmosphère, éliminant des ions tels que Na+ ou Cl- par déposition sur le continent.


La composition chimique des océans n’est pas le résultat d’un apport continu en sel mais résulte d’un équilibre dynamique entre flux d’apport et de consommation.
=> On ne peut pas connaître l’âge de la terre avec ça. ( Sinon elle aurait seulement 13M d’années)




La densité augmente lorsque la température diminue et la salanité augmente.
Leur distribution dans l’océan conduit à une structure verticale stable.


Structure verticale stable:
- Limite les mouvement verticaux. La circulation dans l’océan profond est principalement horizontale.
- les les eaux profondes des changement dans la couche bien mélangée de surface.




Formation d’eau de surface: 2 zones ( Bordure du groenland et mer de Weddel )
Pourquoi ? Dans ces zones, l’évaporation et la formation de glace de mer augmente encore la densité des eaux de surface. Elles peuvent devenir plus denses que les eau de surfaces et plonger vers les fonds océaniques.
=> Circulation thermohaline.


Cycle global de la circulation thermohaline:
Circulation principalement horizontale, les eaux profondes peuvent remonter (graduellement et lentement) vers la surface un peu partout mais surtout dans certaines zones. Elles circulent ensuite vers les hautes latitudes.
=> Générée par les variation de densité des masses d’eau.


Circulation océanique et climat:
Profonde influance sur les températures du globe
Transfert environ la même quantité de chaleur que la circulation atmosphérique.


Toute modification de la circulation thermohaline peut avoir une influence profonde sur le climat.




Chapitre 3b : Systèmes de circulation globaux


Tectonique des plaques: La terre est divisée en différentes plaques rigides qui se déplace les unes par rapport aux autres au cours du temps. (par Wegener)


Cette théorie est construite sur l’observation:
-Présence de fossiles similaires
-Continuité des chaînes de montagnes et glaciers.
-Similitude des côtes.


Pangée: Super-continent formé par l’ensemble des continents avant que ceux-ci ne se séparent. ( séparation il y a 200 millions d’années.) => Théorie de la dérive des continents.


Problèmes de cette théorie : 
* Théorie initialement rejetée car aucun mécanisme physique susceptible d’être le moteur du mouvement des continents ne pu être proposé.
* Comment les continents peuvent ils se frayer un chemin au travers des roches solides des fonds océaniques.


Réponse: Fondée sur arguments géophysique. Ces observations ont conduit entres autres à la théorie de l’expansion des fonds océaniques.




Terre solide: Structure et composition
L’étude des matériaux rocheux de la surface de la terre n’apporte des informations sur la composition chimique et minéralogique que de la partie peu profonde du globe terrestre.


Ce que nous connaissons sur la terre interne est déduit de méthode indirectes ( comme la sismologie ).


Tremblement de terre:
Un tremblement est un évènement durant lequel un mouvement rapide entre deux blocs rocheux résulte en une libération soudaine d’énergie.
Le point d’où l’énergie est émise= Le focus.


Le focus est à 0 - 700 km de profondeur la ou la terre est rigide.
La terre se déforme élastiquement ( elle retrouve sa forme initiale ) jusqu’au point ou la déformation fait apparaître une fracture (faille).


épicentre = Point à la surface de la terre directement au dessus du focus.


Vibrations - Ondes sismiques:
Deux types:
- Ondes P:
        - Ondes primaires ( Pressions )
        - Série de compression et d’expansions
- Ondes S:
        - Ondes secondaires ‘cisaillement)
- Sous forme d’un déplacement perpendiculaire à la direction de propagation 
de l’onde.




Ondes P se propage dans les solides et liquides
Ondes S se propage exclusivement dans les solides car ils ont une rigidité structurelle.


Des informations sur la structure de la terre peuvent être obtenue par tomographie sismique, qui combine plusieur sismographe et qui se abse sur le fait que la vitesse d’une onde dépend de la nature des matériaux qu’elle traverse.


  



4 couches principales: Croute, manteau, noyau externe et le noyau interne.


Transitions:
- Limite Croute-Manteau:
-Discontinuité de Mohorovicic ( ou Moho ) : Zone où la vitesse des ondes sismiques 
augmente rapidement ( Ondes P de 5-6km/s à 8km/s)
Sa profondeur varie sous la croute continentale entre 20 et 70 km.
Sous les océans sa profondeur est de  +- 7km


-En dessous du Moho: La vitesse augmente de manière irrégulière.
Dans le manteau inférieur, l’augmentation est plus régulière.



- Limite Manteau inférieur- Noyau externe:
        - Discontinuité de Gutenberg: 2900km
        Disparition ondes S et décroissance marquée de la vitesse des ondes P
        => Le noyau externe est donc liquide et constitue un fluide métallique.
- à 5150 km de profondeur
        - Vitesse ondes P augmente et ondes S réaparaissent
        => Noyaux interne est solide.




La croute:
Croute continentale plus épaisse, moins dense et plus ancienne que la croute océanique.
pétrologie = Étude de l’origine et évolution de la composition chimique et minéralogique des roches.


Roches composées de minéraux.


3 types: Magmatiques, sédimentaires et métamorphiques.


Roches magmatiques:
Résultent du refroidissement et de la solidification du magma


deux types:
Roches intrusives: Se solidifie lentement en profondeur.(ex: granite)
Roches extrusives: Rejeté par une éruption volcanique puis refroidit brutalement en surface. ( ex : basalte)


deux sous-types:
Les roches felsiques(ex:Granite ou rhyolite),plus riches en Quartz plus claires et moins denses que les roches mafiques(ex: basalte ou gabbro)






	Extrusives
	Intrusives
	Felsiques
	Rhyolite
	Granite
	Mafiques
	Basalte
	Gabbro
	



Roches sédimentaires:
Toutes les roches en surface se décomposent ou s’altèrent en matériaux plus fins => Sédiments.




Lithification = Conversion progressive de sédiments non consolides en roches sédimentaires pour former des strates.


strates = couche homogène d’une roche sédimentaire.


Deux types:
- Grès : formé par grains de sables ( > 63µm  de diamère)
- Shales ou shistes : grains plus fin.


Des roches sédimentaires peuvent se former par procéssus biochimique: Lithification de squelettes et coquilles dans les fond marins => Calcaires.


Roches métamorphiques:
Roche ayant subis une transformation sans procéssus de fusion ( exposées aux hautes températures, hautes pression et/ou fluides actifs chimiquement)


Calcaire => Marbre
Shale => Ardoise
Grès => Quartzite.


Minéraux principaux de la croute:
La croute ( continentale ou océanique) principalement  composé de roches contenant des minéraux silicates ( > 90 %). CAD riche en silice et owygène.


Feldspaths = Minéraux les plus abondant de la croute continentale ( Aluminium, sodium, calcum et potassium)
Quartz = Abondant dans la partie supérieure de la croute continentale.


Les mineraux silicates riches en magnésium et en fer ( ex: Olivines et pyroxène) sont des composé carac. des basaltes de la croute océanique.


Composition de la croute:


Oxygène (46.6%) > Silicon (27.7%) > Aluminium (8.1 %) > Fe  (5%) > Ca (3.6%) > Na (2.8%)


Couverture sédimentaire:
Des sédiments et des roches sédimentaires recouvrent respectivement les basaltes de la croute océanique et les granites/diorites de la croute continentale.


Dans océans: Matière minérale ou biogénique se dépose dans les fonds océaniques sous forme d’un séquence de strates relativement planes.


Roches sédimentaires abondantes dans la croute continentale.
Certaines se forment dans un bassin de sédimentation continental.
Souvent provenance marine,transportées sur le continent par l’activité tectonique.






Le manteau:
Entre la croute et noyau externe
Formé de minéraux silicates qui subissent des transformation structurelles et minéralogiques.


Mineraux les plus abondant: Olivine et pyroxenes présent dans une roche ultramafique nommée péridotite.


Le noyau:
Croute et manteau déficitaires en fer => Sans doute composé au niveau du noyau
Supposition: Noyau fait d’un alliage métallique de Fe-Ni.


Composition terre solide:
Croute océanique:
Basalte
2,9g/cm³
Croute continentale:
Granite
2,7g/cm³
Manteau:
Silicone,Oxygène.
4,5g/cm³
Noyau:
Fer
13g/cm³




Topographie = Configuration d’une surface telle que déterminée par la position et l’élévation de ses caractéristiques.


Chaine de montagne sous-marine (+rift au milieu): Expansion des fonds marins.
Comment le sais-t-on ?
Caractéristique magnétique des roches des fonds marins.
Source du champ magnétique terrestre = noyau


Le champ magnétique n’est pas toujours orienté dans la même direction. Parfois polarité magnétique renversée.
Les mineraux magnétiques de fer present dans la lave sont magnétisé dans la direction du champ magnétique de la terre au moment du refroidissement: Datation radiométrique


Les polarités de ces minerai sont symétrique par rapport à la ride.




Portions de la terre solide soumises à la convection:
- Noyau externe: Enveloppe ou le champ géomagnétique est généré
- Manteau: Force motrice de la tectonique des plaques


Les plaques comprennent la croute et la partie supérieure du manteau
=> Lithosphère : Propriété structurales cassante.(fracture en réponse au stress)


En dessous => Asténosphère:Se comporte comme un fluide(Deforme facilement en réponse au stress)
=> Faible qtté de roche en fusion suffisante pour permettre un mouvement relatif de la croute et du manteau supérieur par rapport au manteau profond.




Dérive des continents et reconstructions paléogéographiques:
Paléogéographie = Etude de la reconstruction de la position des continents dans le passé.
Meilleur outil : Bandes géomagnétiques des fonds océaniques.
        => Limité à environs 200 Ma, l’âge des plus vieilles plaques océaniques encore 
observables.


Méthodes alternatives:
1. Roches sédimentaires:
Indicatives de paléo-latitudes ( = Latitude auxelles les roches se sont formées):
        - Dépot glaciaires ( hautes latitudes)
        - Calcaires coraliens ( latitudes tropicales)
        - Dépotes salins ( latitudes subtropicales, typique de conditions arides )
        - Similitude des fossiles ( indique que les continents étaient joints ou suffisamment 
proches pour permettre la migration des espèces )


2. Magnétisme des roches:
Sur base de l’angle formé par l’orientation du champ magnétique au moment de la formation des roches ( angle 0° = équateur, angle 90° = pôles)




Il n’existe pas de bons indicateurs de paléo-longitutes.


Paléogéographique:
Continents dispersés (500 Ma)
Dérive, rapprochement et collision durant 300 Ma
Pangée: Supercontinent au centre sur l’équateur ( 280 Ma)
Dislocation progressive de la pangée il y a +- 200 Ma.


Plaques tectoniques:
Il en existe environ 20.


Limites de plaques: Typologie
(a) Marges divergentes:
S’écartent et créent des failles
(b) Marges convergentes:
Une passe sous l’autre et création de fosses océaniques et chaînes de montagne
( c ) Marges transformantes
Plaques coulissent.


Moteur de la tectonique des plaques:
Flux = 0,06 W/m² = 60 mW/m²
Au sein du manteau chaleur transporter par convection jusqu’à la base de la lithosphère puis par conduction au travers de cette dernière.


Origine du flux de chaleur: (1) Décroissance radioactive, (2) Chaleur résiduelle datant de la formation de la Terre, (3) Croissance du noyau interne.


Désintégration radioactive:
Element radioactif les plus important au sein de la terre =  Potassium( 40K), l’uranium (235+238U) et le thorium (232Th)
=> Leur désintégration produit un flux de chaleur
A diminuer d’un facteur 5 depuis la formation de la Terre.


Chaleur résiduelle: Lié au réchauffement primitif induit par les collisions entre objets planétaires qui ont conduit à l’accrétion de la terre


Convection mantelique:
Les roches sont ductiles aux condition P et T du manteau.
Lorsqu’elles sont chauffé elles subissent une expansion et deviennent moins denses.
Roche plus froide remplace le manteau qui remonte.


Forces agissant sur les plaques:
Les mouvements résultent de la sommes des forces. ( Gravitationnelle, traction exercée par la section subductée, la force de friction mantelique).


Recyclage de la lithosphère:Altération
Altération physique: Expansion et fracture suite à l’érosion de matériel sus-jacent.
Altération chimique: Dissolution desminéraux.Résidu insoluble ( argile )
…: Erosion et déposition:
Erosion = Processus de transport des produits solides de l’altération vers des bassins ou les sédiements s’accumulent.


… : Accumulation de sédiments et lithification:
Les sédiements sont soumis a des pressions et températures de plus en plus élevées.
=> Subissent compaction et cimentation => Lithification.


…: Surrection:
Collision entre plaques tectoniques => Formation de relief sur les continents
…: Le cycle des roches:
= Régénération complete des roches ( +- 100 millions d’années ).




Craton =Les roches les plus anciennes sur les continents ( il y a 4 milliards d’années)




Le cycle de Wilson:
-Supercontinents se forment et se désagrègent au cours de cycle de 400 à 500 Ma.
- Séquence:
        -Supercontinent se forme à partir de la collision de plus petits.
        - Chaleur s’accumule en dessous conduisant à sa désagrégation
        - Les morceaux dérivent dans toutes les directions
        - Rentrent à nouveau en collisions.


=> Prochain : dans 200 Ma.


Chapitre 4: Introduction à l’analyse des systèmes:
  





Système = Association de différents composants qui intéragissent entre eux.
Composant = réservoir de matière ou d’énergie. Ou un attribut du système.
Exemple: Système climatique inclus: L’atmosphère, les océans, les calottes glaciaires, …


Comment étudier un système:
Identifier les composants et déterminer la natures des intéractions (ou couplage) entre eux




Diagrammes de système: Notation
  





Dans un couplage positif un changement au sein d’un des composants du système stimule au niveau du composant connecté un changement dans la même direction.
Dans un couplage négatif, le changement se fait dans la direction opposée.


Exemples:
  



Un système climatique simplifié: Planete Paquerette
-T° moyenne = 30°C
-Ni nuage, ni océan, ni gaz à effets de serre.
- Sol = grisâtre (absorbe de la lumière )
- Vie = pâquerettes blanches ( réfléchit toute la lumière )
- Soleil = Similaire au notre
- Croissance des pâquerettes = Change avec la température.




Luminosité du soleil dans ce système croit beaucoup plus vite que notre propre soleil.
La vie pourra-t-elle se maintenir ?
La couverture en paquerette et la température moyenne forment un système a deux composants car ils sont interdépendants
1. Réponse de la température de surface par rapport aux changements de couverture en paquerettes:


  

  

( Comme on peu le voir, +- et -+ deviennent - , ++ et -- deviennent +)
 
2.Réponse de la couverture en paquerettes aux changements de température de surface:
-Température minimale et maximale de survie
-Température optimale de croissance.
  

  

3. Etats d’équilibres du système planete paquerette.
  

P1 et P2 = Etats d’équilibre du système
=> Etat dans lequel restera indéfiniment le système si il n’est pas perturbé.




Example:
Nous sommes à l’état P1.
La température monte du coup  la couverture en paquerette augmente, du coup la température descend.
Imaginons que la couverture en paquerette augmente, la température diminue, donc la couverture en paquerette aussi.






4. réponse des états d’équilibre a une perturbation:  Intégrons les boucles de feedback.
Feedback = Mécanisme de changement et de réponse à un changement qui s’auto-perpetue.


On combine les couplages => Système fermé => Boucle de feedback
  



Boucles de feedback négatives: Diminuent les effets de perturbations => Etat d’équilibre stable
Boucles de feedback positives: Amplifient les effets des perturbations => Etat d’équilibre instable


Comment définir le ‘signe’ d’une boucle de feedback ?
Nombre de couplage négatif impair: Feedback négatif
//                pair: Feedback positif




Pour l’équilibre P1:
Feedback négatif, donc aura tendance a maintenir la température et la couverture en pâquerette autour de l’état d’équilibre.
Pour équilibre P2:
Feedback positif, si perturbation,soit se dirigera vers P1 soit vers température au delà du domaine de survie des pâquerettes.




5. Forcage externes: Réponse à l’accroissement de la luminosité solaire.
Forcage externe = Perturbation continue d’un système suite à une action extérieure à celui ci.
=> Le système ne peut retourner vers son état stable original.


-Réponse paquerette à l’augmentation de température: Inchangée
- Réponse température à la couverture en paquerettes:
        Pour une même couverture, la température sera plus haute.


  

La température et la couverture en paquerette sont plus élevé en P1’ qu’en P1


-Augmentation T° sans feedback : DT°0
- Augmentation de température au nouvel équilibre (avec feedback) : DT°eq 
- Contribution ( négative ) due au feedback: DT°f


DT°eq = DT°0 + DT°f et Facteur de feedback : f = DT°eq/DT°0








Les leçons de PP:
Un système n’est pas passif par rapport aux influences internes ou externes.
L’autorégulation est une propriété commune a beaucoup de système naturel.
L’idée d'autorégulation est cohérente avec la théorie de Gaia.


Et sur terre demain ? …
-Effet de serre:
Exemple: CO2 double
        -impact direct sur T° sans feedback:  DT°0+1,2°C
        - Avec feedback: 2 < DT°eq < 5 °C
        => Feedback positifs tendent à amplifier l’effet du au forçage du CO2 .


Chapitre 5: Recyclage des éléments: Le cycle du carbone


Introduction:
Sur terre élément essentiels à la vie tels que C, N, P, S ( des nutriments) sont recyclés en permanence


Carbone organique et inorganique:
Carbone existe sous forme réduite ou oxydée
oxydéé = C combiné avec O
réduit = C combiné avec C,H,N


Carbone organique (réduit 0 ) : CH2O ou C6H12O6
Carbone inorganique:
Oxydé +4 : CO2
Réduite -4: CH4


On exprime les quantités en Gt  = 1 Milliard de tonnes.


Réservoirs de carbone:
  





Cycle du carbone:
  



Cycle carbone organique:
Procéssus court-terme:
Photosynthèse: CO2 + H2O -> CH2O + O2
Conversion du carbone inorganique en carbone organique par les producteurs primaires ( plantes, algues et certaines bactéries). Nécessite un apport en énergie solaire.


Balancée par:


Respiration: Inverse photosynthèse


Décomposition: Consommation de matière organique morte par les micro-organismes.


PP = Production primaire
Hétérotrophe = Consommateurs.


Cycle organique terrestre (pré-industriel):
  





Cycle carbonne organique marin:
Photosynthèse -> Décomposition.


  





Cycle du carbonne organique (long-terme):
- Les procéssus géologiques deviennent des facteurs de contrôle essentiel du CO2 atm. aux échelles de temps longues
- Un faible flux de carbone organique échappe au cycle du carbone à court-terme, il est enfoui dans les sédiments.
        => Au cours du temps: Réservoir gigantesque ( 10.000.000 Gt )


-L’essentiel du carbone organique se trouvent dans des roches à des concentrations n’excédant pas 1%, principalement dans des shales formés par la lithification de boues.
Lorsque la concentration est élevée, l’augmentation de pression et de température à l’enfouissement peut conduire à la formation de pétrole (C marin) ou de charbon ( C continental).


- Ce C est transformé à nouveau en CO2 par altération et oxydation lorsque les roches sont mises en contact avec de l’O2 atmosphérique via le cycle tectonique (uplift).


  







Cycle carbone inorganique:
Court-terme:


Echange air-mer lié à la circulation océanique et productivité marine.
Combien de C dissous dans l’océan: 39.040 Gt(c)




Précipitation, dissolution et déposition de carbonates:
La plus grande partie de CaCO3 provient de la précipitation par les organismes marins ( coquilles, squelettes).
Précipitation abiotique également possible mais négligeable à l’échelle globale.


Ca2+ + 2 HCO3- ⇒ CaCO3 (s) + H2CO3 
-Zone photique ( = zone ou la lumière est disponible ) des océans.
- Sur les fonds, en zone peu profonde.


  





Fermeture du cycle: Altération continentale
- Les roches  en surfaces subissent une altération chimique par des pluies légerement acides.
- Deux grandes classes de roches en surface: Carbonates et silicates. Les carbonates les plus abondants avec C et O = Calcite ( CaCO3 ) et dolomite ( CaMg(CO3)2)
- Pour silicates, ici on se base sur CaSiO3.
- altération chimique peut être résumé par:
        CaCo3 (s) + H2CO3 ⇒ Ca2+ + 2 HCO3- 
        CaSiO3 (s) + 2 H2CO3 ⇒ Ca2+ + 2 HCO3+ + SiO2 + H2O


Puits de CO2 pour le système océan-atmosphère:
- Les produits de l’altérations sont transportés vers les océans ou ils sont utilisés dans les processus biologiques.


- Altération des carbonates et leurs précipitation dans les océans:
        Altération: CaCo3 (s) + H2CO3 ⇒ Ca2+ + 2 HCO3- 
        Précipitation: Ca2+ + 2 HCO3- ⇒ CaCo3 (s) + H2CO3 
=> A long-terme, aucune influence sur la composition chimique du système 
océan-atmosphère.


- Altération des silicates et précipitation carbonates:
        Altération: CaSiO3 (s) + 2 H2CO3 ⇒ Ca2+ + 2 HCO3+ + SiO2 + H2O
        Précipitation: Ca2+ + 2 HCO3- ⇒ CaCo3 (s) + H2CO3 
        Réaction nette : CaSiO3 + CO2 ⇒ CaCO3 + SiO2 


        => Conduit à une transformation nette de CO2(g) en CaCO3 (s). Ce procéssus agit 
comme un puit de CO2 pour le réservoir atmosphérique. (0,03 Gt/an )


Ce flux est faible mais si il n’était pas balancé => Consommation complete du réservoir atm. en 20.000 ans.


  



Flux de retour : Tectonique des plaques => Métamorphisme et volcanisme
sources:
Marges divergentes et convergentes: Injection de CO2 mantelique
Marges convergentes : Transformation des sédiments en roches métamorphiques


Cycle carbone inorganique ( pré industriel):
  





Pourquoi Altération silicates et volcanismes se sont compensés durant l’histoire de la terre ?
Cherchez un mécanisme de feedback.


Intensité volcanisme principalement controlée par le flux de chaleur de la terre interne.
Vitesse d’altération très sensible aux facteurs climatique donc au niveau de CO2.
  

Boucle de feedback négative => Système stable
Cette boucle est le facteur clé du contrôle climatique aux échelles de temps géologiques ( > 106 ans)


Chapitre 6: Evolution du système Terre
1. Spéculations sur le Précambrien:
A. Régulation de climat à long-terme: Paradoxe du ‘Jeune Soleil Faible’ revisité:
Evolution de Ts:
Ts tombe sous 0°C il y a 1,9 milliards d’années et atteint -18°C il y a 4,6 milliards d’années
=> Contradiction avec enregistrements géologiques (eau liquide il y a 3,8 milliards d’années)


Solution ?
Albédo plus faible ? Peu probable
Flux géothermique ? Trop faible
=> Accroissement de l’effet de Serre est le plus probable.


Quel gaz à effet de serre auraient pu être plus abondants par le passé ?
- NH3 ? peu probable car détruit rapidement par rayonnement UV
- CO2 ? probable ( source volcanique importante et puits d’altération limité)
- CH4 ? Probable (source catalysée par l’apparition de la vie et puits limité par l’absence d’oxygène atmosphérique)


Vie, CH4 et composition atmosphérique:
- Méthane peut être produit biotiquement et abiotiquement.
- Micro-organismes méthanogènes pourrait être les premières formes de vie.
- Réactifs inorganiques pour méthanogénese abondants au sein de l’atmosphère primordiale
        CO2 + 4 H2 ⇒ CH4 + 2 H2O
- O2 atm. très faible jusqu’a 2,3 milliards d’années.
- Durée de vie dans une atm. pauvre en O2 beaucoup plus grande qu’aujourd’hui.


B. Premières traces de vie au Précambrien:
Durant premiere moitié 20ème siècle on pensait que vie apparue il y a 540 millions d’année ( macrofossiles = restes d’organismes pluricellulaires)


Récement microfossiles ( = minuscules organismes unicellulaires). Vieux de 3,5 milliards d’années.


Stromatolites:
Consitués de restes de (cyano)bactéries disposés en couches.


Cyanobactéries:
Capable de faire photosynthèse oxygenique et photosynthèse anoxygénique.




Apparition de la photosynthèse oxynenique:
Il y a 2,7 milliards d’années.
Enregistrements géologiques de la présence d’oxygène:
Contenu en O2 atm. était faible avant ~2,3 milliard d’année.
La grande oxygenation(GOE) a eu lieu > 300 millions d’années après l’apparition de la photosynthèse oxygenique.


GOE: - Impact majeur sur le climat
        -Imact majeur sur la biodiversité.


C. Evolution géologique du climat à long-terme:
Le climet est loin d’avoir été stable. Le paléoclimat a varié de manière complexe.
Episodes durant lesquels la surface terrestre était entièrement gelée ( Snowball Earth)


Enregistrement glaciaire à long-terme:
6 périodes glaciaires ( dont 3 précambrien )


Glaciations Paléoproterozoïque ( 2,45 - 2,22 milliards d’années ):
Initié par l’augmentation en CO2 et la réduction simultanée du CH4
=> Période glaciaire de dimension globale


Glaciations Néoproterozoïque (0,75 - 0,6 milliards d’années ):
Sur les 7 continents contemporains.
Réunis en deux super-continents centrés autour de l’équateur.


Terre boule de neige:
Nécessite de réduire les concentration en gaz à effet de serre.
Initiation:
Différents mécanismes possibles ( controverses ).
Exemple: Regroupement des continetns aux basses latitudes => Altération des silicates significative => Consommation du CO2
Catalyse:
  



Retour ‘au chaud’:
- CO2 d’origine volcanique s’accumule dans l’atmosphère car altération des silicates virtuellement nulle sur une terre ‘Boule de neige’.
- L’effet de serre deviens suffisament important pour commencer à faire fondre la glace.
- Le feedback albedo glaciaire-température fonctionne en sens inverse et la callotte glaciaire fond de manière spectaculaire ( en quelque milliers d’années ).
- Température de surfaces augmentent brievement jusqu’à des températures très élevées ( 50 - 60 °C )
- CO2 ensuite rapidement éliminé par l’altération des silicates.


2. Le phanérozoïque:
A. Explosion Cambrienne: Faune D’Ediacara:
Augmentation marquée de l’O2 atm. il y a 542 millions d’années
=> Déclencheur de l’explosion de la vie multi-cellulaire.


La faune d’Ediacara ( australie ) = Contient les tout premier organismes multicellulaires préservé dans les enregistrement géologiques.


Explosion Cambrienne:
Eon phanerozoïque, période à laquelle la vie multicellulaire se développe.
Les organismes aquierent la capacité de secreter des coquilles dures.


Les schistes de Burgess au Canada contiennent la faune laplus célebre du Cambrien.


B. Evolution de l’Oxygène:
C’est l’enfouissement de carbone produit par photosynthèse qui a conduit à une production nette d’oxygène.


Carbonifère(+- 300 Ma) et crétacé ( +- 10 Ma) => Carac. par enouissement important du carbone organique => Niveau d’O2 significativement plus élevé qu’aujourd’hui


Un grand nombre de dépots de charbon exploité provienne du carbonifere.


La taille gigantesque de certains insectes du carbonifère sans doute du à l’enrichissement en O de l’atm.




C. Variation du CO2 et climat:
Variation climatique les plus extrême: Précambrien.
Le climat a également fluctué durant les derniers 540 millions d’années.


Nous sommes dans un bref épisode interglaciaire.


Dans une atm. bien oxygénée, ce sont les variations de CO2 ( causés par tectoniques des plaques, altération des silicates ou la préservation du carbone organique ) qui permettent d’expliquer les fluctuations climatiques.


  



Diminution du CO2:
- Accroissement de l’altération des silicates par :
        - Position équatoriale des continents, topographie.
        - Apparition des plantes vasculaires ( stimule l’altération des roches )
- Préservation accrue du carbone organique


Augmentation du CO2 :
- Diminution de l’altération des silicates (par ex. glaciations, niveau des mers élevé au crétacé)
- Extension rapide des fonds océaniques: CO2 émis par décarbonatation, activité volcanique au niveau des rides.




Climat du Crétacé moyen :
Climat particulièrement chaud.
Des fossiles prouvent que :
        - Alligators en sibérie
        - Dinosaures vivaient au delà du cercle arctique en Alaska.
GEOL-F-101        Université Libre de Bruxelles         /